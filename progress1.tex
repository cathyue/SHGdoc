\documentclass[a4paper,8pt,hyperref, twocolumn]{article}

\title{\bfseries \Large Cavity-enhanced second harmonic generation in a silica whispering-gallery microresonator}
%\author{\normalsize  Xueyue Zhang}
\date{\normalsize \today}

\usepackage[left=16mm,right=16mm, bottom = 16mm, top=20mm]{geometry}
\usepackage{upgreek}
\usepackage{hyperref}
\usepackage{booktabs}
\usepackage{tabularx}
\usepackage{xtab}
\usepackage{graphicx}
\usepackage{listings}
\usepackage{url}
\usepackage{amsmath}



\lstset{
	flexiblecolumns,
	basicstyle = \sffamily,
	keywordstyle = \bfseries,
	commentstyle = \rmfamily\itshape,
	stringstyle = \ttfamily
}

\hypersetup{
colorlinks=false}



%\pagestyle{myheadings}
%\markright{Name: Xueyue Zhang, GTID: 903181650}

\bibliographystyle{unsrt}

\begin{document}

\maketitle

%\section{Introduction}

%[Literature; origin, set-up, phase-matching, advantages: low pump power, continuous wave, possible applications and impacts]

Centrosymmetric materials ... most significant application: surface probe... surface response is intrinsically weak, so several methods are used to enhanced surface SHG (e.g. plasmonic)... cavity boosts the intensity of light, making it a good platform for nonlinear optics...cavity enhanced SHG... Recently, SHG with bare silica cavity (Asano OL)...

Here, second harmonic, originating from symmetry breaking at the surface and bulk multipole response (fig.\ref{pic:Fig1} \textbf{b}), is observed under the continuous wave pump below 1 mW in a WGM microsphere made of centrosymmetric material. 
An unprecedented conversion efficiency of $0.049\%$ W$^{-1}$ benefits from doubly resonant enhancement of ultrahigh $Q$ modes (phase-matching condition[also know as perfect phase matched]), which is achieved by dispersion engineering including thermal effect, optical Kerr effect and geometric control of cavity. 

% Apart from the compensation of cavity mode dispersion[], Additionally, the collecting efficiency of SH signal is significantly increased with the incorporation of a signal tapered fiber. 
The work enriches the nonlinear toolbox of silica/silicon photonics and largely extends the emission range of silica microresonators under low pump power (below $1$ mW), making it possible to push the frequency conversion process down to the quantum regime[refs in Asano OL]. More significantly, the fruitful surface SHG and SFG detection methods can be introduced into (bridged with?) the sensitive microcavity sensing, which enables surface-specific detection with low pump power and high sensitivity.
%The work paves the way for low power and continuous-wave SH sensing. The combination of resonant enhancement and silica surface may enable a series of nonlinear sensitive detection. [BROAD BAND FREQUENCY CONVERSION+SURFACE SENSING]
%\section{Experimental set-up and observation of second harmonic signal}
%[Explain the set-up and how to collect the signal; show the typical spectra and corresponding experimental conditions; explain why other nonlinear processes are absent; f1f2 comparison and conditions]

In the experiment, a silica microsphere (diameter $\sim$ $62$ $\upmu$m) 
is pumped through a tapered optical fiber (waist diameter around $1$ $\upmu$m) at $1550$ nm band \cite{knight1997phase, cai2000observation}, as shown in fig.\ref{pic:Fig1} \textbf{a}. To collect SH signal efficiently, a second fiber taper (waist diameter around 0.5 $\upmu$m) designed for 780 nm band is incorporated into the system. The intrinsic quality factor (Q) for the pumped cavity mode is $4.8\times10^7$. %When the signal tapered fiber is near the microsphere to couple SH signals, the effective intrinsic Q (including the intrinsic loss and loss induced by the signal fiber) decreases and a typical value is $3.6\times 10^7$.%The signals collected by the signal tapered fiber are sent into an electron-multiplying CCD (EMCCD) to extract the spectra.
% The EMCCD and the pump laser are placed at the same side of the microsphere considering the linear momentum conservation requirement in SHG\cite{carmon2007visible, kozyreff2008whispering}.
% A tapered fiber phase-matched at the telecommunication band couples the pump into the microsphere. This tapered fiber is not able to collect the second harmonic signal efficiently due to phase mismatch at around 780 nm and the high radial order of the second harmonic mode\cite{carmon2007visible}. To overcome this problem and collect weak second harmonic signals, another tapered fiber (signal tapered fiber) designed to achieve phase matching condition at second harmonic wavelength is fabricated together with the pump tapered fiber and incorporated into the system as is shown in Fig.\ref{pic:Fig1}a. The tapered fibers can reach critical coupling simultaneously at around 1555 nm and 777 nm. 
% A fiber coupler split 10\% of the pump power from the laser into a power meter to monitor the input power. 
Figure \ref{pic:Fig1}\textbf{c} shows a typical SH spectrum from the electron-multiplying CCD (EMCCD) and the corresponding pump spectrum from the optical spectrum analyzer (OSA). The SH signal of $777.75$ nm appears when pumped at $1555.14$ nm, which deviates only $0.023$\% from the expected wavelength, falling into the resolution tolerance of OSA and EMCCD.
Note that stimulated Raman scattering and parametric oscillation do not occur because their thresholds are above the pump power in the experiment% \cite{spillane2002ultralow, kippenberg2004kerr}. 
Third harmonic generation is also absent due to the phase mismatch in the nonlinear optical process.
% between the mode and its third harmonic modes\cite{carmon2007visible}. 
%Third harmonic signals  collected by the signal fiber are also observable when pumped at some specific modes [need figure?]. 
%[e, ->d]
Moreover, SH signals arise in the full range when cavity modes are pumped from $1545$ nm to $1565$ nm, as shown in Fig.\ref{pic:Fig1}d.
%Thanks to the plethora of modes in the microsphere, SH signals arise in , which is 
Among the occurrence of SH, a maximum signal power of $5$ nW can be obtained via the signal fiber. In comparison of the collecting efficiency of the two fibers, the fiber-cavity coupling is optimized so that the SH signal from the pump fiber is observable but the maximum power is over one order of magnitude weaker than the power from the signal fiber. From either fibers, SH signal is absent when the pump is off-resonance with cavity modes, which helps to eliminate the possibility of spurious signals such as the second order diffraction of the EMCCD grating.
%Using the signal fiber, an SH signal with a power of  (shown in Fig.\ref{pic:Fig1}c) is collected at the pump wavelength of 1561.3 nm. 
%The SH power is calibrated from the EMCCD spectra to represent the power in the signal fiber near the microsphere. 
%In order to compare the collecting efficiency, the pump tapered fiber is connected directly to the EMCCD and the coupling between the pump fiber and the microsphere is optimized to maximize the collected SH power. The SH signal is still observable but the maximum power is only 0.36 nW, which is more than 13 times weaker than the SH power collected by the signal fiber.

%The SH signal is only present when the pump light is in the mode (judging from the transmission). Because of thermal bistability\cite{carmon2004dynamical}, the pump light can be out of the mode when tuning the wavelength to blue side but in the mode when tuning in the opposite direction. At the same wavelength, only the in-mode pump can produce the SH. And the SH signal is always absent when the microsphere is moved far from the pump fiber, which also help to eliminate the possibility of spurious signals created by the second order diffraction of the EMCCD grating.

\begin{figure*}[!ht]
\centering
%\captionsetup{singlelinecheck=no, justification = RaggedRight}
\includegraphics[width=18cm]{Fig1.eps}
\caption{\textbf{Experimental set-up and observation of cavity-enhanced SH signals. a, }The pump light from a tunable laser around 1550 nm is coupled into a silica microsphere through a tapered fiber, and a second fiber is used to collect the SH signal. OSC: oscilloscope. OSA: optical spectrum analyzer. PLC: polarization controller. BS: beam splitter. EMCCD: electron-multiplying CCD. \textbf{b, }SH is generated from the surface dipole response and the bulk multipole response in a WG microsphere. \textbf{c, }Measured SH spectrum (red) and the corresponding pump spectrum (black). \textbf{d, }Measured SH wavelengths versus the corresponding pump wavelengths when different modes are pumped. \textbf{e, }Comparison of SH power collected by signal fiber and pump fiber (10 times magnified).}
\label{pic:Fig1}
\end{figure*}

%The signal tapered fiber in this set-up is critical to collect SH signals efficiently. 


%\section{Thermal effect and Kerr effect assisted phase-matching}
%[Prerequisite phase matching (higher order radial modes) and its problems; P2-P1 relation and how to enhance SH signals; mechanisms for assisted phase-matching; results: dependence on detuning and power; comparison with other SH and silica sphere TH.]

The dependence of SH power on pump power can be derived from coupled mode equations\cite{haus1991coupled}.
\begin{equation}
P_2 = \frac{4|g|^2Q_2\eta_2/\omega_2}{4Q_2^2(2\omega_p/\omega_2-1)^2+1}\frac{16Q_1^2\eta_1^2P_1^2/\omega_1^2}{[4Q_1^2(\omega_p/\omega_1-1)^2+1]^2},
\label{eq:P2P1}
\end{equation}
where the subscripts 1, 2 represent the pumped mode and SH mode respectively. $P_i$ ($i=1, 2$)is the power in the corresponding tapered fiber near the microsphere, $g$ is a coupling coefficient between two modes, which will be looked into in the next section. $Q_i$ is the total quality factor, $\omega_i$ is the mode frequency and $\omega_p$ is the pump frequency. $\eta_{i}=Q_i/Q_{ie}$ is the coupling factor and $Q_{ie}$ is the external quality factor. The pump power depletion is ignored due to the weak second order nonlinear effect in silica. The enhancement of SH power by ultrahigh-Q microresonator is obvious in eq.(\ref{eq:P2P1}). 

Achieving phase-matching condition lies in the heart of the doubly resonant enhancement of second order effects and reaching the high conversion efficiency. 
The ultrahigh Q represents the factor of enhancement but also presents a challenge to phase matching or double resonance in a microresonator\cite{carmon2007visible, kozyreff2008whispering, xu2008second, farnesi2014optical}($\omega_p = \omega_1, 2\omega_p = \omega_2$). SH modes with higher order radial number have been proposed or used to compensate the material and geometric dispersion\cite{kozyreff2008whispering}. 
For SHG, a silica microsphere with a diameter of $60$ $\upmu$m gives rise to good phase-matching between a fundamental mode near $1550$ nm and an SH mode with radial number $q_2=2$ (see Supplementary Information). 
In the experiment, the desired phase-matching can be disturbed by the deviation of cavity from its designed geometry, making one of the mode off-resonance and impeding highly efficient SHG. 
%But due to the discrete distribution of modes, the mode with smallest phase mismatch can reduce the SH power by a factor of $10^{-6}$. This phase matching method is also extremely sensitive to the size of microresonators. A deviation of 3\% in diameter can lead to a reduction of SH power by nearly 4 orders of magnitude. It is difficult to control the size of a microsphere precisely in the experiments. 
Therefore, the cavity dispersion should be finely adjusted to compensate the deviation, where mode frequency shift induced by thermal behavior (namely thermal expansion, thermally induced refractive index change and the optical Kerr effect) can be a versatile tool to achieve the goal\cite{del2011octave, herr2014temporal}.
%Thermal effect and Kerr effect have been utilized to compensate the dispersion in microresonator-based frequency comb generation. These effects can also help to achieve phase matching in SHG. 
%Both of the two effects lead to a red shift of the mode frequency \cite{ilchenko1992thermal, treussart1998evidence,  carmon2004dynamical, fomin2005nonstationary} and there is no need to distinguish them because the focus is steady state continuous wave emission. [NEED IT OR NOT?]

The mechanism of thermal and Kerr assisted phase matching process is explained in fig.\ref{pic:Fig2}a. 
When the pump power is weak and the mode frequency shift is negligible (cold cavity), the pumped and SH modes even with higher radial order usually cannot be on resonance with the pump light and its SH simultaneously. 
Increasing the input power leads to the appearance and broadening of the non-Lorentzian, triangular transmission shape for the pump\cite{carmon2004dynamical}. 
%When the pump power is large enough, the pumped mode shifts more to the red side with increasing intra-cavity power. Tuning the pump frequency from the cold cavity mode to the red side can decrease the detuning $\omega_p-\omega_1$ and increase intra-cavity power, thus making the triangular resonance shape \cite{carmon2004dynamical}. 
The SH mode also experiences a red shift from cold cavity frequency and the rate of red shift is different for the SH of pump light and the SH mode, making it possible for them to be on resonance in the process of tuning the pump frequency. 
At the on-resonance pump frequency for SH, the phase-matching condition is fulfilled and the SH power reaches a peak value (corresponding to state 2 in fig.\ref{pic:Fig2}a and b). 
Due to the ultrahigh Q of the SH resonance, the SH power diminishes rapidly before and after reaching the on-resonance pump frequency (state 1 and 3 in fig.\ref{pic:Fig2}a and b). 
The SH signal is much weaker than the pump so that its thermal and Kerr effects are negligible in this process. 
As shown in fig. \ref{pic:Fig2}c, the SH power is measured by varying the pump wavelength in the detuning range of the gray area in fig. \ref{pic:Fig2}b with a fixed input power of $4.46$ mW. [fitting results here]

\begin{figure*}[!ht]
\centering
%\captionsetup{singlelinecheck=no, justification = RaggedRight}
\includegraphics[width=17cm]{try_ed3.eps}
\caption{\textbf{Thermal effect and Kerr effect assisted phase-matching. a, }Schematic of the phase-matching process. Detuning here is the wavelength relative to the cold-cavity wavelength of the pumped mode (to half of this wavelength for SH detuning). The black (red) arrow represents the detuning of the pump light (its SH). The gray (red) Lorentzian line represents the pumped mode (SH mode). 1-3 show three states with increasing pump wavelength but the same input power. \textbf{b, }Normalized SH power and the pump transmission at different pump wavelength detuning. 1-3 correspond to the three states in panel \textbf{a}. The gray area is enlarged in panel \textbf{c} as the theoretical red line. \textbf{c, }SH power versus pump detuning with the input power of 4.46mW. \textbf{d, }The dependence of maximum SH power at all the pump detuning on the input power.}
\label{pic:Fig2}
\end{figure*}

%When the mode in Fig.\ref{pic:Fig1}b is pumped with an input power of 4.46 mW, the SH power exhibits a clear peak at the pump wavelength of 1555.14 nm, which is shown in fig.\ref{pic:Fig2}c. 
[The red shift of the SH mode is proportional to the intra-cavity power, which is proportional to the pump light detuning from the cold cavity frequency. It means $\Delta \omega_2 = D_{12}\Delta \omega_p$, where $\Delta \omega_2$ is the SH mode frequency shift, $\Delta \omega_p$ is the pump light detuning and $D_{12}$ is the proportionality coefficient.  Using this relation and eq.(\ref{eq:P2P1}), the experimental parameters can be fit by the theory to be $Q_2\times(2-D_{12})=8.57\times 10^5$.]

% [EXPLANATION OF WHY THE INTRACAVITY POWER REMAIN ALMOST UNCHANGED]
%The thermal and Kerr assisted phase matching makes the SH power depend strongly on the pump frequency, which also determines the intra-cavity pump power. At a certain pump frequency (the mode frequency is still at the red side of the pump), when the input power increases, the intra-cavity power also tends to increase, which pushes the mode frequency to the redder side. The pump frequency remains unchanged so that the detuning $\omega_p-\omega_1$ increases to lower the coupling efficiency of the input power into the cavity. Therefore, the intra-cavity pump power increases more slowly than the increase of input pump power. 

The dependence of SH power on power pump is also measured, as shown in fig.\ref{pic:Fig2}d. 
At each input power, we search for the maximum SH output power in the wavelength range from the cold cavity mode (1555.02nm) to the pump on-resonance wavelength. 
Among different values of input power, a critical power manifests itself, at which the SH power reaches the peak value just before the pump light becomes on resonance and then thermally unlocked. 
In this case, the pump light and its SH achieve double resonance simultaneously, which represents the most efficient SHG with the pump power of $879$ $\upmu$W and the conversion efficiency of $0.049\%$ W$^{-1}$.
Below the critical input power, the SH of pump light is off resonance when the pump is still thermally locked in the mode, which largely weakens the SH power.
Above the critical power, the increasing input power shifts the pumped mode more to the red side (the pump is not completely on resonance) and lowers the enhancement of the pump light, which counteracts with the increasing input power and makes the intracavity power almost unchanged at the SH resonant detuning. 
Consequently, the SH power remains the same with increasing input power.
%The dependence of SH power and intra-cavity power on the pump frequency gives rise to a novel relation between the SH power ($P_2$) and the input pump power ($P_1$). When $P_1$ is not large enough for the pump SH to catch the SH mode before the pump frequency catches the mode frequency, the SH mode cannot be on resonance at any pump frequency and the SH power is small because of the ultrahigh Q. When $P_1$ reaches a critical power so that the pump SH and the pump frequency catch the two modes simultaneously, the two denominators in eq.(\ref{eq:P2P1}) reach the smallest value of 1, which gives rise to an efficient SHG. When $P_1$ further increases from the critical power, the intra-cavity power increases slowly so that the SH-on-resonance frequency also moves slowly, therefore, when the pump SH is on resonance with the SH mode, the intra-cavity pump power and the corresponding SH power does not change as rapidly as the input pump power.  

[Fitting]
The critical input power is fitted to be 832.5$\upmu$W and the corresponding SH power is fitted to be 381pW. 
Eq.(\ref{eq:P2P1}) and $\omega_1 = \omega_{10}-B_{11}|\alpha_1|^2$ \cite{carmon2004dynamical}, $\omega_2 = \omega_{20}-B_{12}|\alpha_1|^2$ are used to fit the experimental data, where $\omega_{i0}$ is the mode frequency in the cold cavity, $|\alpha_1|^2$ is normalized to the intra-cavity power of the pumped mode and $B_{1i}$ is the coefficient of intra-cavity power induced frequency shift. The parameters related to the pumped mode can be extracted from the measurements. 

\begin{figure*}[!ht]
%\centering
%\captionsetup{singlelinecheck=no, justification = RaggedRight}
\includegraphics[width=18cm]{Fig3.eps}
\caption{\textbf{SH power histogram with different pump polarization. a, } TM and \textbf{b,} TE modes are pumped to generate SH respectively. Insets: Field amplitude distribution and the direction of the electric field (black arrow).}
\label{pic:Fig3}
\end{figure*}

[where?]
In the THG experiments in silica microresonators\cite{carmon2007visible, farnesi2014optical}, the phase mismatch curve is flat for high order radial modes, and the shift speed of the pump TH frequency and the TH frequency mode may be similar. Consequently, the high order radial modes induced phase matching may play a major role so that the $P_1^3$ relation can be observed. Microresonator SHG in other materials usually use different phase matching strategy to achieve broad band or tunable phase matching, e.g. quasi-phase-matching [cite more later]...The quality factor is also moderate so that the thermal effect and Kerr effect do not manifest themselves in the SHG process.


It is also possible to measure the explicit $P_2 \propto P_1^2$ dependence by introducing another degree of freedom to manipulate the SH mode frequency. 
For example, a control light can be coupled into another mode to change the intra-cavity power and thus achieving the double on-resonance condition at various input pump power. 
The specific measurement plan is beyond the scope of this [letter?] and is still under investigation.


%\section{Origin of second order nonlinearity}



%[Theory for surface \& bulk 2nd nonlinearity; relationship with polarization]

%Silica is a centrosymmetric material in which the electric dipole induced second order nonlinearity is forbidden (see \cite{boyd2003nonlinear} for reference). Surface symmetry broken and bulk multipole become the major sources of second order nonlinearity (see \cite{heinz1991second} for review). 

Apart from the unique power dependence, the SH power enhanced by the microresonator also exhibits a polarization dependence. 
The polarization of the pump light is adjusted so that transverse magnetic (TM) or transverse electric (TE) modes from $1545$ nm to $1565$ nm are pumped. 
The maximum SH power of each SHG process is recorded for the two polarization respectively, as shown in the histogram in \ref{pic:Fig3}. After searching for SH in the wavelength range three times with each polarization, a total number of $69$ ($40$) SHG incidence is recorded for TM (TE) polarization, and the average SH power is $0.843$ nW ($0.619$ nW). 
The dependence on polarization originates from the polarization dependent nonlinear coupling coefficient $g$ in eq.(\ref{eq:P2P1}), which bridges the microresonator with the second order nonlinearity of centrosymmetric material. 

Using the Helmholtz equation and the relation between electric field and nonlinear polarization, the coupling coefficient induced by surface nonlinear response can be derived as
\begin{equation}
g_{s0} = 2i\frac{\omega_1^2}{\omega_2n^2}\frac{\int_{\mathrm{surface} } \mathbf{E}_{02}^*:\upchi^{(2)}_{s0}:\mathbf{E}_{01}\mathbf{E}_{01} \mathrm{d}	\mathbf{S}}{\int |\mathbf{E}_{02}|^2 \mathrm{d}	\mathbf{V}}
\end{equation}
where $n$ is the refractive index, $\upchi^{(2)}_s$ is the surface nonlinear susceptibility, and $\mathbf{E}_{0i}(\mathbf{x})$ is the  normalized electric field so that $\alpha_i\mathbf{E}_{0i}(\mathbf{x})$ represents the complete electric field. 

The bulk multipole nonlinear polarization can be written as $\mathbf{P}_\gamma =  \gamma\nabla(\mathbf{E}\cdot\mathbf{E})$ and $\mathbf{P}_\delta =  \delta(\mathbf{E}\cdot\nabla)\mathbf{E}$, where $\gamma$ and $\delta$ are the nonlinear coefficients. The first term represents a longitudinal wave and can couple with the SH only at the surface. Therefore it can be incorporated into an effective surface susceptibility $\upchi^{(2)}_s = \upchi^{(2)}_{s0}+\upchi^{(2)}_{s,\gamma}$\cite{heinz1991second}. The coupling coefficient induced by the second term can be written as % $a'$
\begin{equation}
g_b =  2i\frac{\omega_1^2}{\omega_2n^2}\frac{\int \delta\mathbf{E}_{02}^* \cdot (\mathbf{E}_{01}\cdot\nabla)\mathbf{E}_{01} \mathrm{d}	\mathbf{V}}{\int |\mathbf{E}_{02}|^2 \mathrm{d} \mathbf{V}}
\label{eq:gb}
\end{equation}
The total coupling coefficient between the two modes is $g = g_s+g_b$. 

There are three non-zero components $\upchi_{\perp \perp \perp}$, $\upchi_{\parallel \parallel \perp}$ and $\upchi_{\perp \parallel \parallel}$ in the surface second order susceptibility tensor of fused silica.  $\upchi_{\perp \perp \perp}$ ($\upchi_{\parallel \parallel \perp}$) plays a major role when TM (TE) mode is pumped. $\upchi_{\perp \parallel \parallel}$ can be ignored in studying SHG due to the non-degeneracy of TM and TE modes. TM modes are preferable in surface induced SHG because $\upchi_{\perp \perp \perp}$ is nearly an order of magnitude larger than $\upchi_{\parallel \parallel \perp}$ \cite{rodriguez2008calibration}. The amplitude of bulk nonlinear response $g_b$ relies on the specific field distribution in the microsphere.  Note that for a TE mode, the divergence is along the polar direction, which exhibits geometric symmetry with regard to the equatorial plane. If both the pumped mode and the SH mode are fundamental in the polar direction (polar number $l$ = azimuthal number $m$), $g_b$ vanishes due to the divergence and the polar symmetry. Because of a better confinement in the radial direction and thus a larger divergence for most of the modes, TM modes tend to have a larger $g_b$ than TE modes. For example, in a silica microsphere with a diameter of 62$\upmu$m, the TM pumped mode with $l_1=m_1=171$ and its SH mode with $l_2=m_2=342$ produces a $g_b$ 18 times larger than that of the TE mode with $l_1-1=m_1=171$ and its SH mode with $l_2-1=m_2=342$ (in both cases, the radial numbers of the pumped mode and SH mode are 1 and 2 respectively due to phase matching considerations). 

The polarization dependence can be utilized to add to the surface specificity for SHG sensing. Mediated by $\upchi_{\parallel \parallel \perp}$, two TE polarized photons can generate a TM polarized photon. For nonlinearity from bulk multipole effects ($g_b$), TE polarized pump can only generate TE SH. In this case, the bulk response can be eliminated and thus restricting the SHG on the surface.



\begin{figure}[!ht]
\centering
%\captionsetup{singlelinecheck=no, justification = RaggedRight}
\includegraphics[width=8cm]{Fig4.eps}
\caption{\textbf{Measured spectra of second-order sum frequency generation (SFG). }The pump light ($\omega_1$) and Raman light ($\omega_2$) are summed to generate the SF signal ($\omega_3$).}
\label{pic:Fig4}
\end{figure}


%[More eg of SH; sum frequency]

Sum frequency generation (SFG) can also take place when the pumped mode produce a Raman signal. 
Shown in fig.\ref{pic:Fig4}b is an SF signal ($804.67$ nm) and the corresponding pump ($1550.88$ nm) and its Raman signal ($1674.22$ nm) with an input power of $7.33$ mW, which is above the Raman threshold for this mode. The deviation of the SF wavelength from the expected value ($804.63$ nm) is much smaller than the resolution of EMCCD.
%The measured wavelengths satisfy the SFG relation $1/1550.88$ $\mathrm{nm} +1/1674.22$ $\mathrm{nm} = 1/804.63$ $\mathrm{nm}$, which is close to the measured SF wavelength 804.67 nm.

[Conclusion]




\bibliography{ref}
\end{document}

