\documentclass[a4paper,12pt,hyperref]{article}

\title{\bfseries \Large Second harmonic generation in silica whispering-gallery microresonators}
%\author{\normalsize  Xueyue Zhang}
\date{\normalsize \today}

\usepackage[left=31.75mm,right=31.75mm]{geometry}
\usepackage{upgreek}
\usepackage{hyperref}
\usepackage{booktabs}
\usepackage{tabularx}
\usepackage{xtab}
\usepackage{graphicx}
\usepackage{listings}
\usepackage{url}
\usepackage{amsmath}



\lstset{
	flexiblecolumns,
	basicstyle = \sffamily,
	keywordstyle = \bfseries,
	commentstyle = \rmfamily\itshape,
	stringstyle = \ttfamily
}

\hypersetup{
colorlinks=false}



%\pagestyle{myheadings}
%\markright{Name: Xueyue Zhang, GTID: 903181650}

\bibliographystyle{unsrt}

\begin{document}

\maketitle

\section{Introduction}

[Literature; origin, set-up, phase-matching, advantages: low pump power, continuous wave, possible applications and impacts]


\section{Experimental set-up and observation of second harmonic signal}
[Explain the set-up and how to collect the signal; show the typical spectra and corresponding experimental conditions; explain why other nonlinear processes are absent; f1f2 comparison and conditions]

In the experiments, a silica microsphere resonator with a diameter of around $62\upmu$m is used to enhance the intensity of light and  consequently, the second order nonlinear effects. A tapered fiber phase-matched at the telecommunication band couples the pump into the microsphere\cite{knight1997phase, cai2000observation}. This tapered fiber is not able to collect the second harmonic signal efficiently due to phase mismatching at around 780nm and the high radial order of the second harmonic mode\cite{carmon2007visible}. To overcome this problem and collect weak second harmonic signals, another tapered fiber (signal tapered fiber) designed to achieve phase matching condition at second harmonic wavelength is fabricated together with the pump tapered fiber and incorporated into the system as is shown in fig.[set-up]. The tapered fibers can reach critical coupling simultaneously at around 1555nm and 777nm. A fiber coupler split 10\% of the pump power from the laser and a polarization controller into a power meter to monitor the input power. The signals collected by the signal tapered fiber are sent into an electron-multiplying CCD (EMCCD) with gratings to extract the spectra. The EMCCD and the pump laser are placed at the same side of the microsphere considering the linear momentum conservation requirement in SHG\cite{carmon2007visible, kozyreff2008whispering}.

Fig.[SHPumpspectra] shows a typical SH spectrum and the corresponding pump spectrum. The SH signal of 777.75$\pm$0.17nm was observed at the pump wavelength of 1555.14$\pm$0.01nm (the uncertainty is limited by the resolution of the spectrum analyzer and the EMCCD), which satisfies the SHG wavelength relation with reasonable deviation. The intrinsic quality factor (Q) for the pumped mode is $4.8\times10^7$. When the signal tapered fiber is near the microsphere to couple SH signals, the effective intrinsic Q (including the intrinsic loss and loss induced by the signal fiber) decreases and a typical value is $3.6\times 10^7$. The SH signal can still be detected at a pump power as low as 882$\upmu $W. Raman scattering and parametric oscillation are absent under this condition because the pump power is below the threshold determined by the mode volume and Q\cite{spillane2002ultralow, kippenberg2004kerr}. Third harmonic (TH) generation does not occur in this mode possibly due to the phase mismatch between the mode and its third harmonic modes\cite{carmon2007visible}. Third harmonic signals  collected by the signal fiber are also observable when pumped at some specific modes [need figure?]. 

The signal tapered fiber in this set-up is critical to collect SH signals efficiently. Using the signal fiber, an SH signal with a power of 5nW (shown in fig.[f1f2]) is collected at the pump wavelength of 1561.3nm. The SH power is calibrated from the EMCCD spectra to represent the power in the signal fiber near the microsphere. In order to compare the collecting efficiency, the pump tapered fiber is connected directly to the EMCCD and the coupling between the pump fiber and the microsphere is optimized to maximize the collected SH power. The SH signal is still observable but the maximum power is only 0.36nW, which is more than 13 times weaker than the SH power collected by the signal fiber.


\section{Thermal and Kerr effect assisted phase-matching}
[Prerequisite phase matching (higher order radial modes) and its problems; P2-P1 relation and how to enhance SH signals; mechanisms for assisted phase-matching; results: dependence on detuning and power; comparison with other SH and silica sphere TH.]

The dependence of SH power on pump power can be derived from coupled mode equations\cite{haus1991coupled}.
\begin{equation}
P_2 = \frac{2}{\omega_2}(\frac{2}{\omega_1})^2\frac{8|g|^2}{4Q_2^2(2\omega_p/\omega_2-1)^2+1}\frac{Q_2Q_1^2\eta_2\eta_1^2P_1^2}{[4Q_1^2(\omega_p/\omega_1-1)^2+1]^2},
\label{eq:P2P1}
\end{equation}
where the subscripts 1, 2 represent the pumped mode and SH mode respectively. $P_i$ ($i=1, 2$)is the power in the corresponding tapered fiber near the microsphere, $g$ is a coupling coefficient between two modes, which will be looked into in the next section. $Q_i$ is the total quality factor, $\omega_i$ is the mode frequency and $\omega_p$ is the pump frequency. $\eta_{i}=Q_i/Q_{ie}$ is the coupling factor and $Q_{ie}$ is the external quality factor. The enhancement of SH power by ultrahigh-Q microresonator is obvious in eq.(\ref{eq:P2P1}). 

The ultrahigh Q also presents a challenge to phase matching, which corresponds to on-resonance conditions ($\omega_p = \omega_1, 2\omega_p = \omega_2$) in a microresonator\cite{carmon2007visible, kozyreff2008whispering, xu2008second, farnesi2014optical}. SH or TH modes with higher order radial number have been proposed or used to compensate the material and geometric dispersion\cite{carmon2007visible, kozyreff2008whispering, farnesi2014optical}. For SHG, a silica microsphere with a diameter of 60$\upmu$m gives rise to good phase-matching between a fundamental mode near 1550nm and an SH mode with radial number $q_2=2$. But due to the discrete distribution of modes, as is shown in fig.[dispersion], the mode with smallest phase mismatch can reduce the SH power by a factor of $10^{-6}$. This phase matching method is also extremely sensitive to the size of microresonators. A deviation of 3\% in diameter can lead to a reduction of SH power by nearly 4 orders of magnitude. It is difficult to control the size of a microsphere precisely in the experiments. Therefore, other mechanisms are required to achieve phase matching. Thermal effect and Kerr effect have been utilized to compensate the dispersion in microresonator-based frequency comb generation\cite{del2011octave, herr2014temporal}. These effects can also help to achieve phase matching in SHG. Both of the two effects lead to a red shift of the mode frequency \cite{ilchenko1992thermal, treussart1998evidence,  carmon2004dynamical, fomin2005nonstationary} and there is no need to distinguish them because the focus is steady state continuous wave emission. 

Fig.[mechanism] explains the mechanism of thermal and Kerr assisted phase matching process. When the pump power is weak and the mode red shift is negligible (cold cavity), the pumped and SH modes even with higher radial order usually cannot be on resonance with the pump light and its SH simultaneously. When the pump power is large enough, the pumped mode shifts more to the red side with increasing intra-cavity power and tuning the pump frequency from the cold cavity mode to the red side can decrease the detuning $\omega_p-\omega_1$ and increase intra-cavity power, thus making the triangular resonance shape \cite{carmon2004dynamical}. The SH mode also experiences a red shift from cold cavity frequency when intracavity power is large. If the speed of red shift is different for the pump light SH and the SH mode, then it is possible for them to be on resonance in the process of tuning the pump frequency. The SH power will reach a peak value at the on-resonance pump frequency. 

When the mode in fig.[spectra] is pumped with an input power of 4.46mW, the SH power exhibits a clear peak at the pump wavelength of 1555.14nm, which is shown in fig.[SHpowerPeak]. The red shift of the SH mode is proportional to the intra-cavity power, which is proportional to the pump light detuning from the cold cavity frequency. It means $\Delta \omega_2 = D_{12}\Delta \omega_p$, where $\Delta \omega_2$ is the SH mode frequency shift, $\Delta \omega_p$ is the pump light detuning and $D_{12}$ is the proportionality coefficient. Using this relation and eq.(\ref{eq:P2P1}), the experimental parameters can be fit by the theory to be $Q_2(2-D_{12})=8.57\times 10^5$.



\section{Origin of second order nonlinearity}
[Theory for surface \& bulk 2nd nonlinearity; relationship with polarization]

[More eg of SH; sum frequency]





\bibliography{ref}
\end{document}

